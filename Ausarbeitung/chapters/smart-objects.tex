\enquote{Interaktion mit Smart Objects}

Anhaltende Fortschritte bei der Miniaturisierung von Mikroelektronik und Sensorik sowie bei Kommunikationstechnologien ermöglichen die Einbettung von vernetzten Kleinstcomputern in Alltagsgegenstände. Solche sogenannten "Smart Things " sind Geräte, die mit einer virtuellen
Präsenz im Internet der Dinge gepaart sind. Sie können miteinander und mit Menschen kommunizieren, ihre Umwelt durch Sensoren wahrnehmen, autonom Entscheidungen treffen und auf die Welt mittels ihrer Aktoren einwirken. Wenn dies im Sinne der menschlichen Benutzer geschieht, agieren vernetzte, schlaue Dinge wie eine unsichtbare Hintergrundassistenz, die unser Leben angenehmer, unterhaltsamer und auch sicherer machen kann.
Um mit solchen Smart Objects umgehen zu können, werden neuartige Benutzerschnittstellen benötigt und In diesem Kapitel werden wir eine davon präsentieren.


\subsection{neu Experiment basiert auf einen generative Ansatz mit semantischen Interaktionsbeschreibungen}


Der Artikel \textit{User interfaces for smart things -- A generative approach with semantic interaction descriptions} \cite{Abe.2015} behandelt die Verwendung  eine neu generative Verfahren um Intuitive Widget zu erstellen sodass es möglich ist , intuitiverweise  mit Smart -Objekte in einer definierte Umgebung (Smart-Home zum Beispiel ) kommunizieren zu können .

\subsubsection{Ziel}

Das Ziel dieser Artikel beziehungweise diese Arbeit ist es, zu untersuchen, wie menschliche Benutzer dabei unterstutzt werden können, sich in Umgebungen zurechtzufinden, die hunderte schlauer Dinge enthalten. In solchen smarten Umgebungen ist es fur Benutzer insbesondere schwierig, jene Geräte, die sie gerade benötigen, effizient aufzufinden und intuitiv auszuwählen, und mit
ihnen in geeigneter Weise zu interagieren. Zudem soll es Benutzern ermöglicht werden, Geräte in derartigen Umgebungen so zu konfigurieren, dass sie in kooperativer Weise Aufgaben erledigen können, welche fur ein einzelnes schlaues Ding zu komplex sind. 


\subsubsection{Konzeptbeschreibung}

Wie bereit erwähnt Um dem automatischen Erzeugung der Benutzerschnittstellen für intelligente Objekte zu ermöglichen , haben  diese Wissenschaftler einen neue  Protokolle in Form eines User Interface Description Language (UIDL) entwickelt  und die Interaktionsgeräte (zum Beispiel Fernbedienungen oder Smartphone) verwenden die Informationen , die durch diese neue Ansatz , um eine geeignete konkrete Schnittstelle auf dem Smartphone Bildschirm der  Benutzer bereitstellen

Auf Basis dieser semantischen Metadaten, welche die Funktionalität von einzelnen Smart Objekt  charakterisieren, wird hier  so ein System vorgestellt, welches ermitteln kann, welche Aufgaben Geräte in schlauen Umgebungen gemeinsam erledigen können. Hierfür gibt der Benutzer lediglich den erwünschten Zielzustand seiner Umgebung an, und das entwickelte  System findet mithilfe diese semantischen Methode Reasoners selbst heraus, ob und wie dieser Zustand erreicht werden kann.
Hierfür stellen letztere Metadaten zur Verfügung, welche die automatische Erzeugung von Nutzungsschnittstellen auf tragbaren Geräten wie Smartphones, Tablets und Smartglasses ermöglichen. Indem schlaue Dinge ihre Interaktionssemantik auf hoher Ebene beschreiben, anstatt nur schnittstellenspezifische Informationen bereitzustellen, ermöglicht das Konzept nicht nur die automatische Erzeugung von grafischen Widgets, sondern gleichzeitig auch gestenbasierte Interaktion sowie Sprachsteuerung. 

Bevor allerdings eine Nutzungsschnittstelle geladen werden kann, um mit einem Intelligenten Objekte in definierte Umgebung  zu interagieren, muss dieser vom Benutzer ausgewählt werden. 
Hierfür wird   aktuelle Technologien aus der optischen Bilderkennung verwendet , Um die Interaktion mit einem Gerät zu initiieren und die mögliche Benutzer  müssen mit diesem Ansatz lediglich mit der Kamera ihres Smartphone oder Tablett auf das Smart-Gerät zielen .


\subsubsection{Bewertung}
Mithilfe der in dieser Artikel entwickelten Technologien können beispielsweise Automatisierungsszenarien zu Hause und in Fabriksumgebungen umgesetzt werden .
dort sollen sich in Zukunft einzelne Geräte oder ganze Fertigungsanlagen automatisch abstimmen und schnell anpassen, um den Produktionsprozess effizienter zu gestalten, insbesondere bei kleinen Losgrößen.

Auf andere Seite optimale Bedingungen ( Indoor oder Outdoor Bedingungen wie zum Beispiel Licht Intensität oder Dunkelheitsrate in dem Raum, wo die Smart Objekte sich befindet  ) müssen unbedingt erfüllt werden um eine korrekte Erfassung von dem Intelligenten Objekt durch Bildverarbeitung verfahren auszuführen. Infolgedessen ist die Korrekte Generierung von  entsprechende Widget auf dem Smartphone  Bildschirm  mithilfe  diese Model basierte Semantische Ansatz  sehr Abhängig  von der korrekte Extraktion von der  Identität (Natur/Form) der Objekt,  mit der Sie interagieren wollen.


